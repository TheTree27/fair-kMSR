\documentclass[a4paper,11pt]{article}


\def \student {Mattis Gerteis}


\def \subject {FPT Approximations for Fair k-Min-Sum-Radii}


\usepackage[utf8]{inputenc}
\usepackage[T1]{fontenc}
\usepackage[ngerman]{babel}
\usepackage{fancyhdr}
\usepackage[top = 2cm, bottom = 2cm, left = 2cm, right = 2cm, headheight = 17pt, includehead, includefoot, heightrounded]{geometry}
\usepackage{parskip}
\usepackage{graphicx}
\usepackage{tikz}
\usepackage{amsmath}
\usepackage{amssymb}
\usepackage{amsthm}
\usepackage{amsfonts}
\usepackage{tabularx}
\usepackage{multirow}
\usepackage{hyperref}
\usepackage{colortbl}
\usetikzlibrary{positioning, automata,arrows}
\usepackage{enumitem}
\usepackage{booktabs}
\usepackage{algorithm}
\usepackage[noend]{algpseudocode}
%%%%%%%%%%%%%%%%%%%%%%%%%%%%%%%%%%%%%%%%%%%%%%%%%%%%%%%%%%%%%%%%%%%%%%%%%%%%%%%%

%%%%%%%%%%%%%%%%%%%%%%%%%%%%%%%%%%%%%%%%%%%%%%%%%%%%%%%%%%%%%%%%%%%%%%%%%%%%%%%%
\fancypagestyle{firstPage}{
  \fancyhf{}
  \renewcommand{\headrulewidth}{0pt}
  \renewcommand{\footrulewidth}{0.5pt}
  \fancyhead{}
  \fancyfoot[R]{\thepage}
}
\fancypagestyle{normalPage}{
  \fancyhf{}
  \renewcommand{\headrulewidth}{0.5pt}
  \renewcommand{\footrulewidth}{0.5pt}
  \fancyhead[L]{}
  \fancyhead[R]{\student
  	}
  \fancyfoot[L]{\subject}
  \fancyfoot[R]{\thepage}
}
\setlength{\parskip}{1em}
\newcommand{\printheader}{
\thispagestyle{firstPage}
\textbf{\subject} \hfill \textit{}\\
\hfill \student \\
 \hfill%
Heinrich-Heine-Universität Düsseldorf \hfill \\[2em]
 \hfill \today \\
\rule{\textwidth}{0.4pt}
\pagestyle{normalPage}
}
%%%%%%%%%%%%%%%%%%%%%%%%%%%%%%%%%%%%%%%%%%%%%%%%%%%%%%%%%%%%%%%%%%%%%%%%%%%%%%%%



\begin{document}
\printheader	
\section*{Guessing Radii from a given k-center solution}

\begin{algorithm}
	\caption{Very Broad Strokes}\label{euclid}
	\textbf{Input:} $k$-center solution $F_0$, largest Radius of k-center solution $r_1^*$, approximation factor $\beta$, $k$, $\epsilon$
	\\
	\textbf{Output:} set $R$ of $O(log^k_{1+epsilon}(k/\epsilon))$ radius profiles
	\begin{algorithmic}[1]
		\State //$\tilde{r}_1=$ as the optimal largest radius is within one interval of:
		\State $lowerBound=(1+\epsilon)^{j-1}*r_1^*/\beta$
		\State $upperBound=F_0k$
		\State step through interval with step size $(1+\epsilon)^{j-1}\frac{F_0}{\beta}\ | \ 1 \leq j < k$
		\ForAll{steps in this interval}
			\State use that step as upper bound and $\frac{\epsilon}{k} * r_n$ as lower bound
			\State find the rest of the intervals as before
		\EndFor
		\State repeat for all subintervals again until a depth of k is reached
	\end{algorithmic}
\end{algorithm}


\begin{algorithm}
	\caption{Medium Granularity}\label{euclid}
	\textbf{Input:} $k$-center solution $F_0$, largest Radius of k-center solution $r_1^*$, approximation factor $\beta$, $k$, $\epsilon$
	\\
	\textbf{Output:} set $R$ of $O(log^k_{1+epsilon}(k/\epsilon))$ radius profiles
	\begin{algorithmic}[1]
		\State //$\tilde{r}_1=$ as the optimal largest radius is within one interval of:
		\State $R_1=\emptyset$
		\State $lowerBound=(1+\epsilon)^{j-1}*r_1^*/\beta$
		\State append $lowerBound$ to $R_1$
		\State $j=1$
		\While{$j < \log_{1+\epsilon}(\beta k)$ and last element of $R_1 < F_0k$}
			\State append $(1+\epsilon)^{j-1}\frac{F_0}{\beta}$ to $R_1$
			\State j++
		\EndWhile
		\State append $F_0k$ to $R_1$
		\State // $R_1$ now contains all guesses for the largest radius
		\State $j=2$
		\ForAll{$r_n$ in $R_1$}
			\State use $r_n$ as upper bound and $\frac{\epsilon}{k} * r_n$ as lower bound
			\State find the rest of the intervals as in line 6-8 and append to $R_{1n}$
		\EndFor
		\State repeat for each new radius to calculate profiles $R_{11}...R_{nml...}$ and increment $j$ until a depth of $k$ is reached
		
	\end{algorithmic}
\end{algorithm}

\begin{algorithm}
	\caption{Exact Algorithm}\label{euclid}
	\textbf{Input:} $k$-center solution $F_0$, largest Radius of k-center solution $r_1^*$, approximation factor $\beta$, $k$, $\epsilon$
	\\
	\textbf{Output:} set $R$ of $O(log^k_{1+epsilon}(k/\epsilon))$ radius profiles
	\begin{algorithmic}[1]
		\State $R=\emptyset$
		\State //$\tilde{r}_1=$ as the optimal largest radius is within one interval of:
		\State $R_1=\emptyset$
		\State $lowerBound=(1+\epsilon)^{j-1}*r_1^*/\beta$
		\State append $lowerBound$ to $R_1$
		\State $j=1$
		\While{$j < \log_{1+\epsilon}(\beta k)$ and last element of $R_1 < F_0k$}
			\State append $(1+\epsilon)^{j-1}\frac{F_0}{\beta}$ to $R_1$
			\State j++
		\EndWhile
		\State append $F_0k$ to $R_1$
		\State // $R_1$ now contains all guesses for the largest radius
		\State $T=$ empty tree with root node
		\ForAll{$r$ in $R_1$}
			\State add $r$ as child to root
		\EndFor
		\While{tree depth $\leq k$}
			\State dfs traversal
			\If{node does not have children \textbf{and} node depth $<k$} 
				\State add $\frac{\epsilon}{k} * r$ as child with $r$ as value of the node
				\State $j=$node depth
				\While{$j < \log_{1+\epsilon}(\beta k)$ and last added child $<r$}
					\State add $(1+\epsilon)^{j-1}\frac{r}{\beta}$ as child
					\State j++
				\EndWhile
			\Else
				\State continue
			\EndIf
		\EndWhile
		\State $R=$ list of dfs searches, with each leaf node reached as entry
		\State return $R$
	\end{algorithmic}
\end{algorithm}




\end{document}
