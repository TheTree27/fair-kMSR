\documentclass[a4paper,11pt]{article}


\def \student {Mattis Gerteis}


\def \subject {FPT Approximations for Fair k-Min-Sum-Radii}


\usepackage[utf8]{inputenc}
\usepackage[T1]{fontenc}
\usepackage{fancyhdr}
\usepackage[top = 2cm, bottom = 2cm, left = 2cm, right = 2cm, headheight = 17pt, includehead, includefoot, heightrounded]{geometry}
\usepackage{parskip}
\usepackage{graphicx}
\usepackage{tikz}
\usepackage{amsmath}
\usepackage{amssymb}
\usepackage{amsthm}
\usepackage{amsfonts}
\usepackage{tabularx}
\usepackage{multirow}
\usepackage{hyperref}
\usepackage{colortbl}
\usetikzlibrary{positioning, automata,arrows}
\usepackage{enumitem}
\usepackage{booktabs}
\usepackage{algorithm}
\usepackage[noend]{algpseudocode}
\usepackage[yyyymmdd]{datetime}
%%%%%%%%%%%%%%%%%%%%%%%%%%%%%%%%%%%%%%%%%%%%%%%%%%%%%%%%%%%%%%%%%%%%%%%%%%%%%%%%

%%%%%%%%%%%%%%%%%%%%%%%%%%%%%%%%%%%%%%%%%%%%%%%%%%%%%%%%%%%%%%%%%%%%%%%%%%%%%%%%
\fancypagestyle{firstPage}{
  \fancyhf{}
  \renewcommand{\headrulewidth}{0pt}
  \renewcommand{\footrulewidth}{0.5pt}
  \fancyhead{}
  \fancyfoot[R]{\thepage}
}
\fancypagestyle{normalPage}{
  \fancyhf{}
  \renewcommand{\headrulewidth}{0.5pt}
  \renewcommand{\footrulewidth}{0.5pt}
  \fancyhead[L]{}
  \fancyhead[R]{\student
  	}
  \fancyfoot[L]{\subject}
  \fancyfoot[R]{\thepage}
}
\setlength{\parskip}{1em}
\newcommand{\printheader}{
\thispagestyle{firstPage}
\textbf{\subject} \hfill \textit{}\\
\hfill \student \\
 \hfill%
Heinrich-Heine-Universität Düsseldorf \hfill \\[2em]
 \hfill \today \\
\rule{\textwidth}{0.4pt}
\pagestyle{normalPage}
}
%%%%%%%%%%%%%%%%%%%%%%%%%%%%%%%%%%%%%%%%%%%%%%%%%%%%%%%%%%%%%%%%%%%%%%%%%%%%%%%%



\begin{document}
\printheader	
\section*{Guessing Radii from a given (fair) k-center solution}

\begin{algorithm}
	\caption{Guessing radii}\label{euclid}
	\textbf{Input:} radii of k-center solution $R_{initial}$, approximation factor $\beta$, $k$, $\epsilon$
	\\
	\textbf{Output:} set $R$ of $O(log^k_{1+\epsilon}(\frac{k}{\epsilon}))$ radius profiles
	\begin{algorithmic}[1]
		\State $R=\emptyset$
		\State $r_1^*=max(R_{initial})$
		\State $j=1$
		\State // find initial candidates for the largest radius and initialize their profiles as sets
		\State append $R_1:=\{(1+\epsilon)^{j-1} * \frac{r_1^*}{\beta}\}$ to $R$
		\While{$j < \log_{1+\epsilon}(\beta k)$ and element of last set added  $< kr_1^*$}
			\State append $R_{j+1}:=\{(1+\epsilon)^j * \frac{r_1^*}{\beta}\}$ to $R$
			\State $j++$
		\EndWhile
		\State append $R_{j+1}:=\{kr_1^*\}$ to $R$
		\State // each profile is then appended with its own possible sub intervals
		\For{$R_j$ in $R$}
			\State $j=2$
			\State $lb=\frac{\epsilon}{k}*R_j[0]$
			\While{$j\leq \log_{1+\epsilon}(\frac{k}{\epsilon})$ \textbf{and} last element of $R_j$ $\leq R_j[0]$}
				\State append $(1+\epsilon)^j * lb$ to $R_j$
				\State $j++$
			\EndWhile
			\State move first element of $R_j$ to the end of $R_j$
		\EndFor
		\State return $R$
	\end{algorithmic}
\end{algorithm}


\end{document}
